\ExplSyntaxOn

%% Input is in \g_tmpa_str
%% Output is in \g_tmpa_bool
\cs_new:Npn \__test_if_string_contains_zero: {
    \str_if_in:NnTF \g_tmpa_str 0 {
        \bool_gset_true:N \g_tmpa_bool
    } {
        \bool_gset_false:N \g_tmpa_bool
    }
}

%% Convert base 10 to bijective base 6.
\NewDocumentCommand \ConvertNumber { m } {
    \str_gclear:N \g_tmpa_str
    \int_gset:Nn \g_tmpa_int {#1}

    %% Split the number into base 6 digits.
    \bool_until_do:nn {
        \int_compare_p:nNn \g_tmpa_int = 0
    } {
        \str_gput_left:Nx \g_tmpa_str { \int_mod:nn \g_tmpa_int 6 }
        \int_gset:Nn \g_tmpa_int { \int_div_truncate:nn \g_tmpa_int 6 }
    }

    %% Convert from base 6 to bijective base 6
    %%
    %% For some ungodly reason, the morons who came up with the LaTeX3
    %% spec didn’t saw a _p version of \str_if_in fit for inclusion into
    %% the language, so we need to do this *the dumb way*.
    \__test_if_string_contains_zero:
    \bool_while_do:nn {
        \g_tmpa_bool
    } {
        \str_compare:nNnTF { \str_head:N \g_tmpa_str } = 0 {
            \str_gset:Nn \g_tmpa_str { \str_tail:N \g_tmpa_str }
        } {
            \int_gset:Nn \g_tmpb_int 1
            \bool_while_do:nn {
                \int_compare_p:n { \g_tmpb_int <= \str_count:N \g_tmpa_str }
            } {
                \int_compare:nNnTF {\str_item:Nn \g_tmpa_str \g_tmpb_int} = {0} {
                    %% For some ungodly reason, setting a character at an index
                    %% is too advanced for this damnable 1970s programming language,
                    %% so we *also* need to *this* *the dumb way*.
                    \str_gclear:N \g_tmpb_str

                    %% Append the part of the string before the digit we need to decrement.
                    \str_gput_right:Nx \g_tmpb_str {
                        \str_range:Nnn \g_tmpa_str 1 {\int_eval:n {\g_tmpb_int - 2}}
                    }

                    %% Append the decremented digit.
                    \str_gput_right:Nx \g_tmpb_str {
                        \int_eval:n {
                            \str_item:Nn \g_tmpa_str {\int_eval:n {\g_tmpb_int - 1}}
                            -
                            1
                        }
                    }

                    %% Append a literal 6.
                    \str_gput_right:Nx \g_tmpb_str {6}

                    %% Append the rest of the string.
                    \str_gput_right:Nx \g_tmpb_str {
                        \str_range:Nnn \g_tmpa_str {\int_eval:n {\g_tmpb_int + 1}} {10000}
                    }

                    %% Overwrite the original string.
                    \str_gset_eq:NN \g_tmpa_str \g_tmpb_str

                    %% And break out of the inner loop.
                    \int_gset:Nn \g_tmpb_int {10000}
                } {
                    %% Do nothing here.
                }
                \int_gincr:N \g_tmpb_int
            }
        }

        \__test_if_string_contains_zero:
    }

    \tl_reverse:N \g_tmpa_str
    \str_use:N \g_tmpa_str
}

\ExplSyntaxOff
