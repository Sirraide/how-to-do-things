%% Sometimes, you may want to parse a box definition and insert
%% content after it; you can already do that with assignments and
%% groups using \afterassignment and \aftergroup, respectively, and
%% we can combine them to accomplish what we want (example and
%% explanation are below):
\newbox\mybox

\def\aftervbox{%
  \afterassignment\aftervbox@aux
  \setbox\mybox\vbox
}

\def\aftervbox@aux{%
  \aftergroup\aftervbox@aux@ii
}

\def\aftervbox@aux@ii{%
  %% Do whatever you want in here with \mybox. In this case, we
  %% just insert its contents twice.
  \copy\mybox
  \box\mybox
}

%%%%%%%%%%%%%%%%%%%%%%%%%%% Example %%%%%%%%%%%%%%%%%%%%%%%%%%%
\aftervbox{foobar}

%% The reason why this works is that the above expands to
\afterassignment\aftervbox@aux\setbox\mybox\vbox{foobar}

%% \afterassignment, when combined with a box assignment, will
%% insert the token passed to it after the opening brace of the
%% box. Thus, we end up with
\setbox\mybox\vbox{\aftervbox@aux foobar}

%% This, in turn, conceptually expands to
\setbox\mybox\vbox{\aftergroup\aftervbox@aux@ii foobar}

%% Next, \aftergroup ends up placing its argument (the token
%% we passed to it in the definition of \aftervbox@aux, that 
%% is, \aftervbox@aux@ii) after the closing brace of the 
%% current group, which in this case is the box definition!
\setbox\mybox\vbox{foobar}\aftervbox@aux@ii

%% Finally, this conceptually expands to.
\setbox\mybox\vbox{foobar}\copy\mybox\box\mybox

%% Which is exactly what we wanted to do: process a box, put it
%% in \mybox, then do something with it. In this case, we end up
%% inserting it twice, so the final result is
foobar
foobar
